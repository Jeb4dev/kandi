%%
%% This is file `minimal_modern.fi.tex',
%% generated with the docstrip utility.
%%
%% The original source files were:
%%
%% uefcsthesis.dtx  (with options: `ex,fi,modern')
%%
%% This is a generated file.
%%
%% Copyright (C) 2018--2022 by Pauli Miettinen <pauli.miettinen@uef.fi>
%%
%% This file may be distributed and/or modified under the conditions of
%% the LaTeX Project Public License, either version 1.3c of this license
%% or (at your option) any later version.  The latest version of this
%% license is in:
%%
%% http://www.latex-project.org/lppl.txt
%%
%% and version 1.3c or later is part of all distributions of LaTeX
%% version 2006/05/20 or later.
%%
%%
%% Tämä on yksinkertainen esimerkki uefcsthesis-luokan käytöstä.
%% Tämä tiedosto tuottaa pro gradu -tutkielman yksipuoleisella asettelulla.
%% Tuottaaksesi PDF-tiedoston, käytä joko lualatex- tai xelatex-ohjelmaa.
%% Esimerkiksi:
%% $ lualatex minimal_modern.fi.tex
%% $ biber minimal_modern.fi
%% $ lualatex minimal_modern.fi.tex
%% Vaihtoehtoisesti voit käyttää latexmk-ohjelmaa:
%% $ latexmk -lualatex minimal_modern.fi.tex
%%
%% Palautettaessa opinnäytetyö kirjastoon mukaan täytyy liittää lähdekoodi, josta
%% palautettava PDF on luoto. Helpointa tämä on jos lähdekoodi on ensin tiivistetty
%% yhteen tiedostoon (+ erilliset kuvatiedostot). Yhdistämiseen voi käyttää latexpand-
%% ohjelmaa (https://www.ctan.org/pkg/latexpand), joka tulee myös TeX Live- ja
%% MiKTeX-jakelupakettien mukana. Esimerkkikäyttö:
%% $ lualatex minimal_modern.en.tex
%% $ biber minimal_modern.en
%% $ latexpand --empty-comments --biber minimal_modern.en.bbl \
%% > minimal_modern.en.tex > flat_thesis.tex
%% $ lualatex flat_thesis
%% $ lualatex flat_thesis
%% joka tuottaa tiedostot flat_thesis.tex ja flat_thesis.pdf, jotka voidaan palauttaa
%% kirjastoon kuvatiedostojen kanssa.
%%
\documentclass[bscthesis,finnish,oneside,biblatex]{uefcsthesis}

%% Korvaa seuraavasta minimal.bib lähdeviitetietokantatiedostosi nimellä.
\addbibresource{library.bib}

%% Korvaa isolla kirjoitetut tekstit omilla tiedoillasi.
%% Työn otsikko, päiväys ja avainsanat täytyy antaa myös englanniksi
\title{Mobiilisovellusten tietoturvatestaus DevSecOps-prosessissa: Menetelmät, työkalut ja vaikutukset} % Työsi otsikko
\title[english]{Mobile Application Security Testing in the DevSecOps Process: Methods, Tools, and Implications} % Otsikko englanniksi
\author{Jesper}{Kauppinen} % Nimesi
\date{\thismonth} % Työsi valmistumiskuukausi ja -vuosi tai \thismonth automaattiseen päiväykseen
%% \date[english]{MONTH YEAR} % Englanninkielistä päiväystä ei tarvita, jos käytät \thismonth-komentoa, mutta se tarvitaan, jos kirjoitat päivän käsin
\city{Kuopio} % Joko Kuopio tai Joensuu
\firstsupervisor{Marko Jäntti} % Ensimmäisen ohjaajan nimi
\secondsupervisor{} % Toisen ohjaajan, jos on, nimi
\keywords{DevSecOps\sep Tietoturvatestaus\sep Mobiilisovellus\sep CI/CD} % Avainsanat erotetaan \sep-komennolla
\keywords[english]{DevSecOps\sep Security Testing\sep Mobile Application} % Avainsanat englanniksi

%% ACM:n CCS-luokittelun LaTeX-komennot saa luotua ACM:n työkalulla osoitteessa
%% https://dl.acm.org/ccs/ccs.cfm
%% Kopioi työkalun tuottama LaTeX-koodi tähän (alun XML-koodia ei tarvitse
%% kopioida). Esimerkiksi:
%% \ccsdesc[500]{Some Class}

\begin{document}
\maketitle
\begin{abstract}
KIRJOITA SUOMENKIELINEN TIIVISTELMÄSI TÄHÄN
\end{abstract}

\begin{abstract}[english]
KIRJOITA ENGLANNINKIELINEN TIIVISTELMÄSI TÄHÄN
\end{abstract}

\frontmatter
\tableofcontents
\mainmatter

\chapter{Johdanto}
\label{cha:johdanto}

Tutkimuskysymykset:

\begin{description}
    \item[Pääkysymys:] Miten mobiilisovellusten tietoturvatestaus voidaan integroida osaksi jatkuvaa integraatiota ja toimitusprosessia DevSecOps-prosessissa, ja mitkä ovat tähän liittyvät parhaat käytännöt?
    \begin{description}
        \item[Alakysymys 1:] Mitkä ovat manuaalisen ja automatisoidun tietoturvatestauksen edut ja rajoitukset mobiilisovellusten DevSecOps-prosessissa?
        \item[Alakysymys 2:] Mitä työkaluja käytetään automatisoituun ja manuaaliseen mobiilisovellusten tietoturvatestaukseen DevSecOps-prosessissa?
        \item[Alakysymys 3:] Miten DevSecOps-prosessi vaikuttaa mobiilisovellusten käyttöönottoon ja ylläpitoon?
    \end{description}
\end{description}

Biblio Authoreiden siivous:
Etunimen eka kirjain -> etunimi

Tutkimuksen rakenne

Johdannon alku:
motivoidaan lukija
Muut aikaisemmat tutkimukset: miten 5 aikaisempa artikkelia, viittauksia lähteisiin
Miksi tutkimusta tarvitaan? Ei aikaisempaa tutkimusta tähän specifiin aiheeseen
minun työn tavoitteet: pääcontribuutio
Työn rakenne: mitä luvut käsittelevät (1 kappale)

\chapter{Teoria}
\label{cha:teoria}

\section{Johdanto teoria-lukuun}
\label{sec:teoria-johdanto}

Tässä luvussa määritellään käsitteet kuten DevSecOps, CI/CD, tietoturvatestaus, mobiilisovellus, jne.

% Perustelu luvun rakenteelle ja käsitteille; OWASP-dokumenttien rooli grey literature -aineistona.

%--- 2  DevSecOps & CI/CD-putki --------------------
\section{DevSecOps ja CI/CD}
\label{sec:devsecops-cicd}
  \subsection{DevOps-mallin evoluutio DevSecOpsiin}
  \subsection{CI/CD-arkkitehtuuri}
  \subsection{Kontti- ja orkestrointialustat}

%--- 3  Tietoturvatestaus --------------------------
\section{Tietoturvatestaus}
\label{sec:testing}
  \subsection{SAST, DAST, IAST ja SCA}
  \subsection{Testauksen integrointi putkeen}

%--- 4  Haavoittuvuustaksonomiat ja standardit -----
\section{Haavoittuvuustaksonomiat ja standardit}
\label{sec:taxonomies}

  \subsection{OWASP Top 10 -sarja}
    \subsubsection{Web-sovellukset (2021)}   % \cite{owasp-top10-web-2021}
    \subsubsection{API-rajapinnat (2023)}     % \cite{owasp-top10-api-2023}
    \subsubsection{Mobiilisovellukset (2024)} % \cite{owasp-top10-mobile-2024}
    \subsubsection{CI/CD-putket (2023)}       % \cite{owasp-top10-cicd-2023}

  \subsection{Common Weakness Enumeration (CWE)}
  \subsection{Common Vulnerability Scoring System (CVSS v3.1)}
  \subsection{Kartoitus OWASP - CWE - CVSS}

%--- 5  Turvallinen koodaus ------------------------
\section{Turvallinen koodaus}
\label{sec:secure-code}
  \subsection{Periaatteet}
  \subsection{Ohjeistot ja cheat-sheetit}
  \subsection{Automatisoidut tarkistukset}

%--- 6  Mobiilisovellusten erityispiirteet ---------
\section{Mobiilisovellusten erityispiirteet}
\label{sec:mobile}
  \subsection{Android vs.\ iOS turvallisuusmallit}
  \subsection{Kehitysmallit ja teknologiat}
  \subsection{Mobile Top 10 -riskien soveltaminen}

%--- 7  Yhteenveto -------------------------------
\section{Yhteenveto}
\label{sec:teoria-yhteenveto}
% Tiivistää keskeiset käsitteet ja nivoo ne tutkimuskysymyksiin.

%-------------------------------------------------
\chapter{Menetelmät}
\label{cha:methods}

Tässä luvussa kuvataan, miten kirjallisuusaineisto kerättiin ja analysoitiin.
Menettely perustuu yksinkertaistettuun kirjallisuuskatsaukseen, ei täyteen
PRISMA- tai Kitchenham-protokollaan.

Tämän työn tavoitteena on vastata neljään tutkimuskysymykseen:

\section{Tutkimuskysymykset ja tavoitteet}
\label{sec:rq}
\begin{itemize}
  \item \textbf{RQ1}: Miten mobiilisovellusten tietoturvatestaus voidaan integroida osaksi jatkuvaa integraatiota ja toimitusprosessia DevSecOps-prosessissa, ja mitkä ovat tähän liittyvät parhaat käytännöt?
  \item \textbf{RQ2}: Mitkä ovat manuaalisen ja automatisoidun tietoturvatestauksen edut ja rajoitukset mobiilisovellusten DevSecOps-prosessissa?
  \item \textbf{RQ3}: Mitä työkaluja käytetään automatisoituun ja manuaaliseen mobiilisovellusten tietoturvatestaukseen DevSecOps-prosessissa?
  \item \textbf{RQ4}: Miten DevSecOps-prosessi vaikuttaa mobiilisovellusten käyttöönottoon ja ylläpitoon?
\end{itemize}
% Selitä lyhyesti, miksi systemaattinen kirjallisuuskatsaus on tarkoituksenmukainen (TODO).
% Perustele SLR-valinta (Kitchenham 2009).

\section{Protokolla}
\label{sec:protocol}

Erillistä, virallisesti rekisteröityä katsausprotokollaa ei laadittu,
koska kyseessä on kandidaatintutkielma. Hakustrategia ja kriteerit
määriteltiin iteratiivisesti työn aikana ja esitetään alla.


\section{Hakustrategia}
\label{sec:search-strategy}

\subsection{Tietokannat}
Hakukanaviksi valittiin \emph{IEEE Xplore}, \emph{Google Scholar}
ja \emph{EBSCOhost}, koska ne kattavat sekä vertaisarvioidut
IT-alan artikkelit että harvakseltaan indeksoidut kirjat.

\subsection{Hakuvaihe 1: laajat haut}
Taulukko~\ref{tab:first-phase} esittää ensimmäisen vaiheen
hakulauseet ja tulosmäärät. Tulokset olivat paikoin kymmeniä tuhansia,
minkä vuoksi hakua ei seulottu käsin tässä vaiheessa.

%-------------------------------------------------
\begin{table}[htbp]
  \centering
  \footnotesize  % hieman pienempi fontti mahtumisen helpottamiseksi
  \caption{Ensimmäisen vaiheen hakulauseet ja tulosmäärät.}
  \label{tab:first-phase}
  \begin{tabularx}{\textwidth}{l>{\ttfamily\small\raggedright\arraybackslash}X l r}
    \toprule
    \textbf{ID} &
    \textbf{Hakulause} &
    \textbf{Tietokanta} &
    \textbf{Tuloksia} \\
    \midrule
    1.1 & DevSecOps                                                                                                & IEEE Xplore     & 148     \\
    1.2 & ("Continuous Integration" AND "Continuous Delivery") OR "CI/CD"                                          & IEEE Xplore     & 542     \\
    1.3 & (security OR cybersecurity) AND "Mobile App*" AND (integrat* OR implement*)                              & IEEE Xplore     & 2\,262  \\
    2.1 & automat* AND manual testing AND Security                                                                  & IEEE Xplore     & 804     \\
    3.1 & Security Testing Tools                                                                                   & IEEE Xplore     & 8\,516  \\
    3.2 & Security Testing Tools for Mobile Apps                                                                   & Google Scholar  & 267\,000 \\
    4.1 & security testing AND impact (deployment OR maintenance)                                                  & IEEE Xplore     & 398     \\
    4.2 & (DevSecOps OR DevOps OR CI/CD) AND impact AND (deployment OR maintenance)                                & IEEE Xplore     & 62      \\
    4.3 & DevOps                                                                                                   & EBSCOhost       & 208     \\
    4.4 & DevSecOps Impact                                                                                         & Google Scholar  & 3\,420  \\
    4.5 & Mobile Security DevOps                                                                                   & Google Scholar  & 16\,800 \\
    \bottomrule
  \end{tabularx}
\end{table}

\subsection{Hakuvaihe 2: tarkennetut haut}
Esivalinnan jälkeen hakulauseita rajattiin. Esimerkiksi
\texttt{"DevSecOps" AND mobile} (EBSCOhost) palautti vain yhden osuman,
mikä helpotti manuaalista seulontaa. Täydellinen lista rajatuista hauista
on liitteessä~\ref{app:queries}.

% Preambeliin samat paketit (jos eivät jo ole)

\begin{table}[htbp]
  \centering
  \footnotesize
  \caption{Toisen vaiheen tarkennetut hakulauseet ja tulosmäärät.}
  \label{tab:second-phase}
  \begin{tabularx}{\textwidth}{l>{\ttfamily\small\raggedright\arraybackslash}X l r}
    \toprule
    \textbf{ID} &
    \textbf{Rajattu hakulause} &
    \textbf{Tietokanta} &
    \textbf{Tuloksia} \\
    \midrule
    1.1  & Integrating Continuous "DevSecOps" Security Testing Techniques                                                            & IEEE Xplore    & 4  \\
    1.2  & Implement* AND Security CI/CD Pipeline* NOT cloud AND Development AND DevSecOps                                           & IEEE Xplore    & 3  \\
    1.3  & 2020 $\rightarrow$ ("security testing" OR "cybersecurity testing") AND "Mobile App*" AND (integrat* OR implement*) AND NOT ("Education*" OR "JavaScript") & IEEE Xplore    & 6  \\
    2.1a & Comparative Analysis of Automated and Manual Testing for Cybersecurity                                                    & IEEE Xplore    & 2  \\
    2.1b & Automated "Security Testing" (DevSecOps OR DevOpsSec) AND CI/CD NOT IoT                                                   & IEEE Xplore    & 4  \\
    3.1  & Integrating Continuous Security Testing Tools (DevSecOps OR CI/CD) AND "Case Study" NOT "data driven"                     & IEEE Xplore    & 4  \\
    3.2  & Automated Security Testing "Mobile Apps" Tools                                                                            & Google Scholar & 1  \\
    4.1  & Impact of DevOps to deployment and maintenance                                                                            & IEEE Xplore    & 1  \\
    4.2  & Impact Analysis in DevOps based Software Development Quality Deployment NOT (Monolith OR "Big data")                      & IEEE Xplore    & 3  \\
    4.3a & DevSecOps AND mobile                                                                                                      & EBSCOhost      & 1  \\
    4.3b & DevOps AND mobile applications AND Continuous Integration and Deployment                                                  & EBSCOhost      & 1  \\
    4.4  & "DevSecOps practices and their impact"                                                                                    & Google Scholar & 1  \\
    4.5  & "Mobile DevOps" AND "Security Testing" AND "CI/CD"                                                                        & Google Scholar & 3  \\
    \bottomrule
  \end{tabularx}
\end{table}

\subsection{Ajallinen ja kielirajaus}
\label{sec:search-limits}

Hakukielet rajattiin suomeen ja englantiin. Käytännössä kaikki sisään
otetut artikkelit ovat englanninkielisiä (n = 33), yksi suomenkielinen työ
sisältyy aineistoon. Aikarajausta ei asetettu, mutta
valitut julkaisut sijoittuvat luontevasti vuosille 2016–2025.

\section{Julkaisujen valintaprosessi}
\label{sec:selection}

Kaikki tarkennettujen hakujen (hakuvaihe 2) osumat ladattiin
kokoteksteinä ja arvioitiin seuraavilla sisäänotto-/poissulkukriteereillä:

\begin{itemize}
  \item \textbf{Include}: käsittelee DevSecOpsia, CI/CD:tä, mobiilisovellusten tietoturvatestausta, mobiilisovellusten DevOps.
  \item \textbf{Exclude}: ei kokotekstiä saatavilla, pelkkä esitys/abstrakti,
        kieli muu kuin suomi tai englanti, tai julkaisu ei käsittele mitään
        seuraavista: turvallisuus, CI/CD, mobiili. Pääpaino IoT tai OT.
\end{itemize}

Tämän prosessin tuloksena valittiin \textbf{n = 34} artikkelia analyysiin.

\cite{ozdenizci2024mobilizing} kokoteksti saatiin pyytämällä lukuoikeutta.

%-------------------------------------------------
\subsection{Datan poiminta}
\label{sec:data-extraction}

Viitteet vietiin taulukkoon, johon kerättiin kentät
ID, vuosi, tutkimusmenetelmä, otsikko, tekijät, DOI/URL ja tietokanta
(Taulukko~\ref{tab:extractionfields}). Poiminnan teki yksi henkilö;
kaksoispoimintaa tai formaalia laatuluokitusta ei tehty.


\begin{table}[htbp]
  \centering
  \footnotesize
  \caption{Käytetty poimintalomake.}
  \label{tab:extractionfields}
  \begin{tabular}{p{3cm}p{8.5cm}}
    \toprule
    \textbf{Kenttä} & \textbf{Kuvaus} \\
    \midrule
    ID           & Viitteen yksilöllinen tunniste (Haku ID + juokseva nro) \\
    Vuosi        & Julkaisuvuosi (4 numeroa) \\
    Tutkimus\-menetelmä & Artikkelin itse raportoima menetelmä (esim.\ case-tutkimus, survey) \\
    Otsikko      & Artikkelin otsikko sellaisenaan \\
    Tekijät      & Kirjoittajien etunimen ensimmäinen kirjain ja sukunimi \\
    Linkki       & DOI-osoite tai pysyvä URL \\
    Tietokanta   & IEEE Xplore, Google Scholar tai EBSCOhost \\
    \bottomrule
  \end{tabular}
\end{table}

%-------------------------------------------------

\section{Työkalut ja toistettavuus}
\label{sec:tools}
Overleaf, Github

%-------------------------------------------------
\section{Synteesi}
\label{sec:synthesis}

\subsection{Kuvaileva analyysi}
Julkaisut jaoteltiin julkaisuvuoden ja tietokannan mukaan
(Taulukko~\ref{tab:descriptive}). Tulosten jakauma osoittaa, että
suurin osa lähteistä on ilmestynyt 2023–2024
ja painottuu IEEE Xploreen.

\subsection{Sisällön tiivistys}
Artikkeleista poimittiin esimerkkihavaintoja, jotka ryhmiteltiin
viiteen pääteemaan (Taulukko~\ref{tab:themes}).
Tarkempaa temaattista koodausta ei tehty kandidaatintutkielman
rajausten vuoksi.

\begin{table}[htbp]
  \centering
  \footnotesize
  \caption{Julkaisujen jakautuminen vuosittain ja tietokannoittain.}
  \label{tab:descriptive}
  \begin{tabular}{lrrr}
    \toprule
    \textbf{Vuosi} & \textbf{IEEE Xplore} & \textbf{Google Scholar} & \textbf{EBSCOhost} \\
    \midrule
    2016 & 1 & 0 & 0 \\
    2018 & 0 & 0 & 1 \\
    2019 & 1 & 0 & 1 \\
    2020 & 4 & 0 & 0 \\
    2021 & 1 & 0 & 0 \\
    2022 & 2 & 0 & 0 \\
    2023 & 9 & 3 & 0 \\
    2024 & 7 & 5 & 0 \\
    \midrule
    Yhteensä & 25 & 8 & 2 \\
    \bottomrule
  \end{tabular}
\end{table}

%-------------------------------------------------
\section{Poikkeamat alkuperäisestä suunnasta}
\label{sec:deviations}

Hakua jatkettiin huhtikuuhun 2025 saakka ja
hakulauseita tarkennettiin toistuvasti tulosjoukkojen koon perusteella.
Muutoin tutkimuskysymykset ja kriteerit pysyivät muuttumattomina.


%======================================================================
\chapter{Tulokset}
\label{cha:tulokset}

%----------------------------------------------------------------------
\section*{Luvun rakenne}
Tässä luvussa esitetään kirjallisuuskatsauksen empiiriset havainnot
neljän tutkimuskysymyksen (RQ1–RQ4) mukaisesti.  Menetelmä- ja
hakuprosessi kuvattiin luvussa~\ref{cha:menetelmat}; johtopäätökset
käsitellään luvussa~\ref{cha:analyysi}.

%----------------------------------------------------------------------
\section{Kirjallisuushaun yleiskuva}

Kirjallisuushaku tuotti yhteensä 283\,360 viitettä eri tietokannoista (IEEE, Google Scholar, EBSCOhost) ensimmäisessä vaiheessa.
Hakulauseiden tarkentamisen ja toisen vaiheen haun jälkeen seulonta kohdistettiin 29 relevanttiin artikkeliin ja 2 kirjaan.
Seulontaprosessissa poistettiin duplikaatit ja epäolennaiset julkaisut. Lopulliseen tarkasteluun valikoitui 26 teosta.
Kuva~\ref{fig:prisma} esittää PRISMA-kaavion hakuprosessin etenemisestä ja seulontavaiheista.


\begin{figure}[h]
  \centering
%  \includegraphics[width=.8\textwidth]{figures/prisma.pdf}
  \caption{PRISMA-kaavio kirjallisuushaun vaiheista}
  \label{fig:prisma}
\end{figure}

\begin{table}[htbp]
  \centering
  \footnotesize
  \caption{Lähteiden kattavuus tutkimuskysymyksittäin}
  \label{tab:rq_matrix}
  \begin{tabular}{p{5cm}cccc}
    \toprule
    \textbf{Lähde} & \textbf{RQ1} & \textbf{RQ2} & \textbf{RQ3} & \textbf{RQ4} \\
    \midrule
    Rangnau 2020                                     & X   & X   & X   & --  \\
    Marandi 2023                                     & X   & X   & X   & X   \\
    Nikolov 2024                                     & X   & X   & X   & --  \\
    Afifah 2024                                      & X   & (X) & (X) & --  \\
    Putra 2022                                       & X   & X   & X   & (X) \\
    Kushwaha 2024                                    & X   & (X) & (X) & (X) \\
    Feio 2024                                        & X   & X   & X   & --  \\
    Schneider 2020                                   & X   & X   & X   & --  \\
    Baheux 2023                                      & X   & (X) & X   & --  \\
    Grasselli 2023 (Digital Twin)                    & --  & --  & X   & (X) \\
    Mukti 2023 (Password Manager)                    & --  & --  & X   & --  \\
    Kohli 2020 (COVID Tracer apps)                   & --  & X   & X   & --  \\
    Zhu 2024 (SAST tools)                            & --  & --  & X   & --  \\
    Rane 2024 (Auto vs Manual scan)                  & --  & X   & X   & --  \\
    Raman 2019 (SQLi AutoScript)                     & X   & X   & X   & --  \\
    Göttel 2023 (IEC 62443)                          & X   & X   & X   & --  \\
    Aljohani 2023 (Unified framework)                & --  & X   & X   & --  \\
    Dupont 2021 (Incremental CC)                     & X   & X   & X   & --  \\
    Nutalapati 2023 (Tools \& Techniques)             & --  & X   & X   & --  \\
    Saleem 2023 (Survey, Pakistan)                   &      &     &     &     \\
    Chatterjee 2024 (Operational Efficiency)         &      &     &     &     \\
    Savor 2016 (Facebook CD)                         &      &     &     &     \\
    Offerman 2022 (DevOps Practices)                 &      &     &     &     \\
    Hsu 2019 (Practical Security Automation, book)   &      &     &     &     \\
    Rohin 2018 (Mobile DevOps, book)                 &      &     &     &     \\
    Chung 2024 (DevSecOps Metrics)                   &      &     &     &     \\
    \bottomrule
  \end{tabular}
\end{table}


Taulukosta~\ref{tab:rq_matrix} nähdään, että neljä päälähdettä
kattavat laajimmin RQ1–RQ3-alueet, mutta RQ4:ää tukevia tutkimuksia on
vähiten.

\subsection{Lähdeanalyysi}
\begin{description}
    \item[\cite{rangnau2020_cst}] Artikkelissa esitellään tapaustutkimus, jossa integroidaan kolme dynaamista tietoturvan testausmenetelmää(WAST, SAS ja BDST) Docker- ja GitLab-pohjaiseen jatkuvan integraation ja toimituksen (CI/CD) putkeen. Tutkijat määrittelevät kahdeksan keskeistä vaatimusta DevSecOps-käytännön onnistuneelle käyttöönotolle, kuten nopean rakennusajan (alle 10 minuuttia), rinnakkaiset testiajojen mahdollisuudet sekä selkeät haavoittuvuusraportit. Tutkimuksen tulokset osoittavat näiden tavoitteiden pääosin toteutuneen. Käytännön haasteina nousivat esiin erityisesti konttien hallintaan ja testitapausten konfigurointiin liittyvät ongelmat, joihin tarjotaan alustavia ratkaisuehdotuksia. Tutkimus vastaa tutkimuskysymyksiin RQ1–RQ3, jotka liittyvät tietoturvatestausten integraatioon,työkalujen vertailuun sekä automatisoidun ja manuaalisen testauksen eroihin. Tutkimuksen rajoitteina ovat keskittyminen vain web-sovellusten testaukseen (ei mobiilisovelluksiin) sekä painotus dynaamisiin testeihin ilman staattisten testien (SAST) vertailua. Artikkeli tarjoaa kehitystiimeille käytännönläheisiä suosituksia DevSecOps-käytäntöjen tehokkaaseen käyttöönottoon.
\end{description}

\begin{description}
    \item[\cite{marandi2023_ias}] Artikkelissa esitellään tapaustutkimus, jossa Docker-konttikuvien tietoturvaskannaus automatisoidaan GitHub Actions -pohjaisessa CI/CD-putkessa yhdistämällä Snyk (staattinen analyysi, SAST) ja StackHawk (dynaaminen testaus, DAST)build- ja test-stagioissa. Tulokset osoittavat, että SAST/DAST-yhdistelmä havaitsi varhaisessa kehitysvaiheessa merkittävästi enemmän haavoittuvuuksia sekä lyhensi skannauksen ja korjauksen kokonaiskestoa verrattuna manuaalisiin prosesseihin. Lisäksi reaaliaikainen dashboard-raportointi paransi kehittäjien näkyvyyttä turvallisuustilanteeseen ja nopeutti reagointia. Artikkeli vastaa RQ1: CI/CD-putken integroinnin kuvaus, RQ2: automaation edut manuaaliseen testaamiseen nähden,RQ3: työkalujen (Snyk ja StackHawk) valinta ja konfigurointi sekä RQ4: vaikutus käyttöönottoon ja ylläpitoon proaktiivisen valvonnan kautta. Laadunarvioinnissa vahvuuksina korostuvat selkeä case-toteutus ja syvä integraatio CI/CD-putkeen, kun taas rajoituksina mainitaan testauksen kohdistuminen vain yhteen web-sovellukseen (rajoitettu yleistettävyys mobiiliympäristöihin)ja manuaalisen testauksen vertailun pinta-alaisuus.
\end{description}

\begin{description}
    \item[\cite{nikolov2024_fit}] Artikkelissa esitellään tapaustutkimus, jossa uhkamallinnus integroidaan Jenkins-pohjaiseen CI/CD-putkeen automatisoidusti OWASP Threat Dragon -työkalulla. Uhkatiedot tallennetaan MySQL-tietokantaan ja niitä hyödynnetään OWASP ZAP -työkalulla tehtävässä kohdennetussa tietoturvatestauksessa. Tulokset osoittavat, että automatisointi parantaa testauksen tehokkuutta hidastamatta kehitystä. Haasteina nousevat esiin työkalujen integroinnin monimutkaisuus ja automatisoinnin rajoitteet. Artikkeli tarjoaa ratkaisuksi kehyksen tietojen automaattiselle viennille ja integroinnille osaksi DevOps-prosessia. Rajoitteina ovat mobiilisovellusten ulkopuolelle jättäminen ja kvantitatiivisen arvioinnin puute. Tutkimus vastaa RQ1: integraatio Jenkins CI/CD-putkeen, RQ2: 3 haastetta (integroinnin vaikeus, uhkamallinnuksen automatisointi ja testaustekniikoiden yhdistäminen) ja RQ3: käytetyt työkalut (Threat Dragon, MySQL, OWASP ZAP).
\end{description}

\begin{description}
    \item[\cite{afifah2024_coi}] Artikkelissa esitellään tapaustutkimus, jossa kehitetään GitHub Actions -pohjainen CI/CD-putki, joka automatisoi PHP-sovellusten lähdekoodin obfuskoinnin ja salauksen (Blowfish-algoritmilla), sekä integroi staattisen (SonarQube, SAST) ja dynaamisen(OWASP ZAP, DAST) tietoturvatestauksen. Tutkimuksen tulokset osoittavat, että lähdekoodin obfuskointi suojaa tehokkaasti koodin kaappauksilta ja vaikeuttaa sen ymmärtämistä hyökkääjän näkökulmasta. Lisäksi havaittiin, että obfuskointi lisää CI/CD-prosessin suoritusaikaa maltillisesti. Käytännön haasteina nousivat esiin työkalujen integrointi CI/CD-workflow’hun ja suorituskykyerojen mittaaminen eri sovellustyypeillä, joihin ratkaisuina esitetään valmiita Docker-kuvia ja -konfiguraatioita. Tutkimus vastaa pääkysymykseen RQ1 kuvaamalla, miten tietoturvatestit ja koodisuojaus voidaan automatisoida DevSecOps-prosessissa. RQ2 ja RQ3 saavat epäsuoraa vastausta käytettyjen työkalujen ja vaikutusten kautta, mutta manuaalitestausta tai mobiilisovelluksia ei tarkastella. Tutkimuksen rajoitteina ovat konteksti rajoittuen PHP-websovelluksiin, manuaalisten testausmenetelmien puuttuminen sekä mobiiliympäristöjen ulkopuolelle jääminen.
\end{description}

\begin{description}
    \item[\cite{putra2022_devsecops}]Putran ja Kabetan (2022) tapaustutkimuksessa kehitetään ja otetaan käyttöön DevSecOps-malli ketterään kehitysprosessiin integroimalla automatisoituja tietoturvatestejä (SAST, DAST) CI/CD-putkeen verkkopohjaiselle Node.js/Express.js-sovellukselle ja Flutter-mobiilikäyttöliittymälle. Tutkimuksessa toteutettiin GitLab CI/CD -putki AWS EC2:lla, jossa Docker-kontit, GitLab Runner sekä työkalut Njsscan (SAST) ja OWASP ZAP(DAST) mahdollistivat nopean rakennus-, testaus- ja käyttöönottoprosessin. Automatisointi lyhensi julkaisusyklin 2–3 tunnista 3–4 minuuttiin ja paljasti haavoittuvuuden (CWE-327)commit-hetkellä. RQ1:een vastattiin eriyttämällä pipeline-vaiheet,käyttämällä kontteja ja ”fail-fast”-periaatetta, parhaiksi käytännöiksi nousivat automaattinen testaus enenn julkaisua ja salaisuuksien hallinta GitLab-muuttujilla. RQ2 ja RQ3 käsittelivät pääasiassa automatisoidun testauksen etuja, kuten nopeus, jatkuva testaus sekä käytettyjä työkaluja, jättäen manuaalisen testauksen rajat ja mobiilikohtaiset haasteet (esim. sovelluskauppajakelu) sivuun. RQ4:n osalta prosessi nopeutti käyttöönottoa ja vähensi ylläpitokustannuksia, vaikka mobiilispesifiä vaikuttavuutta ei kvantifioitu. Tutkimuksen vahvuuksiin kuuluu selkeä ja systemaattinen tekninen toteutus ja mittaustulokset, mutta rajoituksina ovat kapea teknologiavalikoima (Node.js/Flutter), rajallinen vertailu turvallisuusmetriikoissa sekä mobiilikontekstin pinnallinen käsittely,vaikka Flutter mainitaankin.
\end{description}

\begin{description}
    \item[\cite{kushwaha2024_cct}]Artikkelissa kuvataan Azure DevOps-ympäristöön rakennettu konttipohjainen DevSecOps-putki, jossa uhkamallinnus (Microsoft TMT) sekä automatisoidut SAST-, DAST- ja SCA-vaiheet (Codacy, OWASP ZAP, GitHub Advanced Security/CodeQL) on integroitu CI/CD-työnkulkuun Docker-kuvien kokoamisesta tuotantoon; kohteena oli haavoittuvuuksilla siemenetty finanssialan web-sovellus, jossa uhkamallinnus paljasti 26 STRIDE-riskiä, Codacy 28 staattista ongelmaa, GHAS 16 koodivirhettä ja 3 kriittistä riippuvuutta, ZAP raportoi XSS- ja SQL-injektiohälytyksiä, ja kaikki skannaukset valmistuivat muutamassa minuutissa. RQ1: Integraatio toteutuu YAML-pohjaisena monivaiheisena putkena, joka käynnistyy jokaisesta commitista; parhaat käytännöt ovat shift-left-filosofia, konttien yhtenäiset rakennusvaiheet ja salaisuuksien skannaus. RQ2: Automaatio vähentää manuaalista työtä ja tarjoaa jatkuvan kattavuuden, mutta vaatii tulosten manuaalista priorisointia ja täydentäviä penetraatiotestejä. RQ3: Mainitut työkalut ovat Codacy, OWASP ZAP, GHAS/CodeQL ja Microsoft TMT; erillisiä manuaalisia mobiilityökaluja ei käsitellä. RQ4: Putki nopeuttaa sovelluksen käyttöönottoa, pienentää tuotantoriskejä ja helpottaa ylläpitoa, mutta mobiilisovelluksia ei tarkastella. Vahvuudet: realistinen finanssi-case, integroitu Azure DevOps-ratkaisu ja kvantitatiiviset haavoittuvuustulokse. Rajoitteet: web-keskeinen rajaus ilman mobiilikontekstia, tarkkojen skannausaikojen puuttuminen ja vertailun rajoittuminen yhteen CI-alustaan.
\end{description}

\begin{description}
    \item[\cite{feio2024_empirical}] Artikkelissa esitellään empiirinen tapaustutkimus, jossa rakennetaan DevSecOps-kehys. Kahdeksanvaiheinen CI/CD-putki (Plan–Monitor) laajennetaan jatkuvalla turvallisuustestauksella, integroimalla työkaluja kuten SonarQube (SAST), OWASP Dependency-Check (SCA) ja OWASP ZAP (DAST) Jenkinsiin, mikä mahdollisti automaattisen haavoittuvuuksien tunnistamisen. Tuloksena havaittiin 1795 haavoittuvuutta, joista 47 kriittisiä, ja kehittäjät arvioivat työkalujen hyödyllisyyttä myönteisesti. RQ1: Integraatio toteutui laajentamalla DevOps-putkea uusilla testausvaiheilla ja automatisoiduilla työkaluilla. Parhaat käytännöt sisälsivät avoimen lähdekoodin työkalujen käytön sekä jatkuvan palautteen keräämisen. RQ2: Automatisoidut työkalut antavat jatkuvan kattavuuden ja nopeat tulokset, mutta eivät löydä kompleksisia logiikkavirheitä, kun taas manuaaliset (esim. penetraatiotestaus) vaativat erikoisosaamista ja puuttuivat pilotista. RQ3: Työkaluina käytettiin yleisimpiä SAST-, DAST- ja SCA-työkaluja, kuten SonarQube, OpenVAS ja ZAP. RQ4: Artikkelissa ei tarkasteltu mobiilisovelluksia, mutta DevSecOpsin käyttöönotto paransi haavoittuvuuksien tunnistusta ja kehittäjien tietoturvatietoisuutta. Laadullisesti tutkimuksen vahvuuksia olivat realistinen teollinen ympäristö, kvantitatiiviset tulokset, kehittäjäpalaute sekä työkalujen käytännön integrointi. Rajotteisiin kuuluivat mobiilikontekstin puute ja joidenkin pipeline-vaiheiden (käyttöönotto, operointi) toteuttamatta jääminen.
\end{description}

\begin{description}
    \item[\cite{schneider_2020_fuzzing}]
    Tapaustutkimus kuvaa automatisoidun fuzzing-pohjaisen tietoturvatestauksen käyttöönottoa turkkilaisen Kuveyt Türk -pankin Mobil Şube -mobiilisovelluksen taustapalveluihin. Keskeisenä työkaluna käytettiin Fuzzino-kirjastoa integroituna olemassa olevaan Citrus- ja Apache HttpClient -pohjaiseen testikehikkoon, joka ajetaan DevOps-putkessa WCF-rajapintoihin kohdistuvina HTTP-pyyntöinä.

    Yhden AllBadStrings-heuristiikan avulla generoitiin yli 50 000 fuzzattua pyyntöä, joista keskimäärin 61\,\% saavutti palvelun, 27\,\% torjuttiin WAF-palomuurissa ja 12\,\% johti virhevastaukseen ilman suoraan hyödynnettäviä haavoittuvuuksia, mutta paljastaen syötedatan validointitarpeita.

    Tutkimus vastaa tutkimuskysymyksiin (RQ) seuraavasti:
    \begin{itemize}
        \item \textbf{RQ1:} Integraatio toteutettiin lisäämällä Fuzzino-kirjasto olemassa olevaan testausinfrastruktuuriin.
        \item \textbf{RQ2:} Automatisoinnin etuna nähtiin kyky vastata kasvaviin vaatimuksiin ja parantaa haavoittuvuuksien löytämisen tehokkuutta.
        \item \textbf{RQ3:} Artikkelissa mainitaan työkalut Fuzzino, Citrus, Apache HttpClient, Appium, Selenium sekä WAF-ratkaisu ja pankin omat 2FA-palvelut.
        \item \textbf{RQ4:} DevSecOps paransi testauksen jatkuvuutta ja palautteen nopeutta, mutta käyttöönoton nopeuteen ei havaittu merkittävää vaikutusta.
    \end{itemize}

    Tutkimuksen vahvuuksina korostuivat realistinen pankkiympäristö, kvantitatiiviset tulokset ja integraation yksityiskohtaisuus. Rajoitteina mainittiin teknologiakattavuuden ja mobiiliratkaisujen rajallisuus.
\end{description}

\begin{description}
    \item[\cite{baheux2023_droidsectester}] esittelevät DroidSecTester-työkaluketjun, joka integroi kontekstia hyödyntävän mallipohjaisen tietoturvatestauksen Android-sovelluksiin nelivaiheisena DevSecOps-putkena. Työkaluketjuun kuuluvat myös FlowDroid data-analyysiin ja VPatChecker haavoittuvuuksien tunnistamiseen. Todettuna tuloksena esitetään, että GHERA-testialustalla vähintään 38 \% haavoittuvuuksista voitiin mallintaa ja tunnistaa nykyisellä haavoittuvuusmallin määrittelyllä.

    Tutkimus vastaa tutkimuskysymyksiin (RQ) seuraavasti:
    \begin{itemize}
        \item \textbf{RQ1:} Integraatio toteutuu nelivaiheisena prosessina, jossa luodaan sovelluksen malli, rikastetaan sitä tietovirroilla FlowDroidilla, generoidaan testejä ConTestilla ja tunnistetaan haavoittuvuuksia VPatCheckerillä. Parhaat käytännöt liittyvät mallipohjaisen testauksen hyödyntämiseen tehokkuuden saavuttamiseksi ja kontekstin huomioimiseksi.
        \item \textbf{RQ2:} Tutkimus painottuu automaatioon
        \item \textbf{RQ3:} Artikkelissa mainitaan FlowDroid, ConTest ja VPatChecker automatisoituun testaukseen.
        \item \textbf{RQ4:} Työkalun tavoitteena on auttaa kehittäjiä löytämään haavoittuvuuksia kehityksen aikana ja parantaa Android-ekosysteemin tietoturvaa, eikä se suoraan käsittele käyttöönottoa.
    \end{itemize}

     Vahvuuksina voidaan nähdä DroidSecTester-työkaluketjun toteutus ja kvantitatiiviset tulokset tunnistettujen haavoittuvuuksien määrästä. Rajoitteita ovat vertailun puute muihin työkaluihin haavoittuvuuksien tunnistustarkkuudessa ja Android-keskeisyys, joka rajoittaa yleistettävyyttä.
\end{description}

\begin{description}
    \item[\cite{grasselli2023_digitaltwin}] esittelevät NFV-MANO-viitekehykseen pohjautuvan menetelmän, joka mahdollistaa automaattisen digitaalisen kaksosen (DT) käyttöönoton ja hallinnan älykkäiden ajoneuvojen kyberturvallisuustestaukseen. Digitaalista kaksonen toimii kyberharjoitusalueena, jossa voidaan turvallisesti kokeilla potentiaalisia hyökkäyksiä ja vastatoimia virtuaalisessa ympäristössä. Keskeisiksi työkaluiksi mainitaan Open Source MANO (OSM) orkestraattorina, OpenStack pilvi-infrastruktuurin hallintaan sekä Kubernetes konttien hallintaan. Tutkimuksessa toteutettiin käyttötapaus, jossa DT:n avulla toteutettiin onnistunut verkkoskannaus simuloimalla infotainment-järjestelmän altistumista (eng. compromising) Bluetoothin kautta, osoittaen mitkä laitteet olivat näkyvissä haavoittuvan sovelluksen kautta. Tuloksena havaittiin, että digitaalinen kaksonen mahdollistaa joustavan ja nopean testausympäristön luomisen, mikä nopeuttaa turvatoimien kehittämistä ja validointia.

    Tutkimus vastaa tutkimuskysymyksiin (RQ) seuraavasti:
    \begin{itemize}
        \item \textbf{RQ1:} -
        \item \textbf{RQ2:} -
        \item \textbf{RQ3:} Työkaluina automaattiseen testaukseen mainitaan OSM, OpenStack, Kubernetes ja cansniffer1
        \item \textbf{RQ4:} Automatisoidun digitaalisen kaksosen havaittiin parantavan nopeuttavan riskianalyysiä ja tuotantoon vientiä ja ylläpitokatkoksia, mutta vaikutuksia mobiilisovelluksiin ei tarkastella.
    \end{itemize}

    Vahvuuksia ovat käytännönläheinen arkkitehtuuritoteutus todellisilla työkaluilla (OSM, OpenStack), realistinen käyttötapaus ajoneuvon verkkojen simulointiin sekä joustavuuden demonstrointi. Rajoituksiin kuuluvat mobiilikontekstin puute, kvantitatiivisten tulosten vähäisyys sekä rajaus telekom-alan NFV-MANO-viitekehykseen, mikä rajoittaa yleistävyyttä.
\end{description}

\begin{description}
    \item[\cite{mukti2023_passwordmanager}] Artikkeli esittelee RootPass-salasanan hallintasovelluksen, joka on toteutettu Flutterilla ja yhdistää RootBeer-kirjaston kahdeksaan heuristiikkaan perustuvan juuritunnistuksen (eng. root detection) sekä turvamekanismit: AES-256-salauksen, PBKDF2-avainjohdannuksen, Flutter Secure Storagen ja sormenjälkitunnistuksen.

    Turvallisuutta mitattiin MobSF-penetraatiotestauksessa (60/100, Low Risk) ja ImmuniWeb-testillä OWASP Mobile Top 10 riskejä arvioitaessa. Tulosten mukaan RootPass saavutti korkeimman turvallisuuspisteet (60/100) verrattuna Keeperiin, LastPassiin ja Dashlaneen sekä vähemmän OWASP-riskejä (12 vs. 16–19). Root-testissä sovellus sulkeutui roottatuilla laitteilla 100 \%:n tarkkuudella.

    Tutkimus vastaa tutkimuskysymyksiin (RQ) seuraavasti:
    \begin{itemize}
        \item \textbf{RQ1:} -
        \item \textbf{RQ2:} -
        \item \textbf{RQ3:} Artikkelissa mainitaan automatisoituja työkaluja kuten MobSF ja ImmuniWeb Mobile App Security Test.
        \item \textbf{RQ4:} -
    \end{itemize}

     Vahvuuksia tutkimuksessa ovat realistinen Android-prototyyppi ja kvantitatiivinen vertailu tunnettuja sovelluksia vastaan. Rajoituksia ovat keskittyminen vain Androidiin ja root tunnistukseen, eikä varsinaisesti tietoturvatestaukseen tai DevSecOps-näkökulmaan.
\end{description}

\begin{description}
    \item[\cite{kohli2020_covid_tracer}] Artikkelissa tutkitaan neljän Android-pohjaisen COVID-19 -kontaktinjäljityssovelluksen tietoturvaa keskittyen staattiseen analyysiin ja potentiaalisten hyökkäyspintojen tunnistamiseen. Analyysissä hyödynnettiin ensisijaisesti MobSF-työkalua koodin staattiseen tarkasteluun ja Drozeria hyökkäysrajapinnan kartoittamiseen. Keskeiset tulokset osoittivat useita OWASP Mobile Top 10 ja CWE -haavoittuvuuksia testatuissa sovelluksissa.

    Tutkimus vastaa tutkimuskysymyksiin (RQ) seuraavasti:
    \begin{itemize}
        \item \textbf{RQ1:} -
        \item \textbf{RQ2:} Automatisoitu MobSF havaitsee noin 70 \% haavoittuvuuksista ja tarvitsee manuaalista varmistusta sekä täydentävää dynaamista testausta.
        \item \textbf{RQ3:} Artikkeli mainitsee MobSF:n, Drozerin, ADB:n, apktoolin, Memu-emulaattorin ja suunnitellun mutta käyttämättä jääneen Burp Suiten.
        \item \textbf{RQ4:} -
    \end{itemize}

     Artikkelin vahvuuksiin kuuluvat case-toteutus todellisilla sovelluksilla ja kvantitatiivisten tulosten esittäminen, kun taas rajoituksia ovat testauksen painottuminen staattiseen analyysiin ja rajattuminen Androidiin sekä kontaktinjäljityssovelluksiin verrattuna laajempaan mobiilisovelluskehitykseen.
\end{description}

\begin{description}
    \item[\cite{zhu2024_sast_tools}] Artikkeli keskittyy staattisten sovellusturvallisuustestaus (SAST) työkalujen vertailuun Android-ympäristössä. Tutkijat kehittävät VulsTotal-alustan, joka yhdistää 11 SAST-työkalua ja normalisoi niiden haavoittuvuusraportit. Tutkimuksessa luodaan 67 standardoitua haavoittuvuustyyppiä, joita testattiin synteettisillä (GHERA, MSTG\&PIVAA) ja CVE-pohjaisilla benchmarkeillä.

    Konkreettisia tuloksia olivat työkalujen vaihteleva haavoittuvuuskattavuus (korkein 67\%, alin 22\%) ja tehokkuus eri vertailuaineistoilla, sekä tavukoodiin perustuvien työkalujen nopeampi suorituskyky.

    Tutkimus vastaa tutkimuskysymyksiin (RQ) seuraavasti:
    \begin{itemize}
        \item \textbf{RQ1:} -
        \item \textbf{RQ2:} -
        \item \textbf{RQ3:} Mainitaan ja arvioidaan 11 automatisoitua SAST-työkalua Androidille.
        \item \textbf{RQ4:} -
    \end{itemize}

    Tutkimuksen vahvuuksia ovat yhtenäisen arviointialustan luominen, kattava 11 työkalun vertailu, laaja empiirinen aineisto sekä realistiset testausympäristöt. Rajoituksena keskittyminen vain Android SAST-työkaluihin
\end{description}

\begin{description}
    \item[\cite{rane2024_analysis_scanning}] vertailevat automatisoidun haavoittuvuus skannauksen ja manuaalisen penetraatiotestauksen tehokkuutta verkkosivustojen tietoturva-analyysissä. Keskeisiä tarkasteltuja työkaluja ovat automatisoidut skannerit kuten Netsparker ja penetraatiotestauksessa käytetyt työkalut kuten Burp Suite ja Nmap. Tutkimuksessa todettiin, että manuaalinen penetraatiotestaus on tarkempi ja tehokkaampi haavoittuvuuksien löytämisessä kuin automatisoitu skannaus, joka tuottaa usein vääriä positiivisia tuloksia.

    Tutkimus vastaa tutkimuskysymyksiin (RQ) seuraavasti:
    \begin{itemize}
        \item \textbf{RQ1:} -
        \item \textbf{RQ2:} Manuaalisen testauksen vahvuuksia ovat tarkkuus ja kyky havaita monimutkaisia haavoittuvuuksia, kuten käyttöoikeuksien korotus ja epäsuora objektiviittaus. Rajoituksena on testauksen vaatima suuri työmäärä. Automaation etuja ovat nopeus, resurssitehokkuus sekä toistettavuus, mutta sen heikkouksina korostuvat virhepositiivisten tulosten riski ja analyysin pinnallisuus.
        \item \textbf{RQ3:} Työkaluihin kuuluivat automatisoiduille Netsparker/Invicti sekä manuaalisille Burp Suite, SQLmap, Metasploit ja Nmap.
        \item \textbf{RQ4:} -
    \end{itemize}

     Tutkimuksen vahvuuksia ovat käytännönläheinen vertailu todellisella verkkosivustolla, kvantitatiiviset tulokset haavoittuvuuksien määristä sekä selkeä metodologinen jako automatisoitujen ja manuaalisten menetelmien välille. Rajoituksina yksittäiseen verkkopalveluun testaus, CI/CD-integraation puute ja mobiilinäkökulman sekä pitkäkestoisen seurannan puuttuminen.
\end{description}

\begin{description}
    \item[\cite{raman2019_sqli}] artikkelissa esitellään ja verrataan automatisoitua SQLiAutoScript-menetelmän manuaaliseen lähestymistapaan laajamittaisten SQL-injektioiden penetraatiotestien hallinnassa web-sovelluksissa. Keskeisenä työkaluna menetelmässä hyödynnetään sqlmapia yhdistettynä komentosarja- ja lokitustoimintoihin, testausympäristöinä toimivat Windows batch- ja Linux Bash -komentosarjat.

   Vertailussa manuaaliseen testaukseen 24 kokeen otoksessa menetelmä suoritti 110 970 HTTP-kyselyä (noin 2813 \% enemmän), vähensi virheitä 66,7 \%, lyhensi yksittäisen testin ajankulun 30 \% ja lisäsi todennettuja löydöksiä 21 \%.

    Tutkimus vastaa tutkimuskysymyksiin (RQ) seuraavasti:
    \begin{itemize}
        \item \textbf{RQ1:} Automatisoitu scriptaus mahdollistaa testien jatkuvan suorittamisen, mikä tukee epäsuorasti CI/CD-integraatiota.
        \item \textbf{RQ2:} Automatisoitu testaus vähentää virheitä, parantaa skaalautuvuutta ja säästää aikaa, kun taas manuaalinen testaus on altis ihmisvirheille ja hidastaa suurten testimäärien käsittelyä.
        \item \textbf{RQ3:} Työkaluina käytettiin sqlmap:ia sekä automatisoiduissa että manuaalisissa testeissä.
        \item \textbf{RQ4:} -
    \end{itemize}

    Vahvuuksia ovat kvantitatiiviset tulokset (esim. HTTP-pyynnöt, virhemäärät), realistinen testausympäristö sekä selkeä vertailu manuaalisen ja automatisoidun lähestymistavan välillä. Rajoituksiin kuuluvat keskittyminen vain web-sovelluksiin (ei mobiilia), rajattu työkalukattavuus (vain sqlmap) ja keskittyminen SQL injektioiden testaukseen.
\end{description}

\begin{description}
    \item[\cite{gottel2023_validating}] esittävät kvalitatiivisen analyysin IEC 62443-4-2 -standardin mukaisten teollisuusautomaation ja ohjausjärjestelmien (IACS) komponenttien kyberturvavaatimusten automatisoidusta validoinnista DevSecOps-prosessissa. Analyysi kartoittaa standardin komponenttivaatimukset (CR:t) ja liittää ne CI/CD-putken vaiheisiin käyttämällä valko-, harmaa- ja mustalaatikkotestausta. Kirjoittajat tunnistavat CR-kohtaisesti sopivat kaupalliset työkalut sekä avoimen lähdekoodin vaihtoehdot.

    Tuloksena syntyy kattavuusmatriisi, joka osoittaa, että suurin osa mustalaatikkotesteistä voidaan automatisoida, mutta valkolaatiikkotestauksen kattavuus on heikko. Useiden CR-kohtien validointi vaatii edelleen manuaalista työtä.

    Tutkimus vastaa tutkimuskysymyksiin (RQ) seuraavasti:
    \begin{itemize}
        \item \textbf{RQ1:} Parhaat käytännöt: shift-left ja kaupallisten sekä avointen ratkaisuiden yhdistäminen.
        \item \textbf{RQ2:} Artikkeli käsittelee joitain automaation etuja ja manuaalisen testauksen tarvetta.
        \item \textbf{RQ3:} Keskeisiä työkaluja ovat CRT- ja VIT-sertifioidut kaupalliset fuzzerit ja skannerit (Acheron, Achilles, Defensics, Nessus, Raven ES, VHunter) sekä avoimen lähdekoodin ratkaisut kuten AFL++, Burp Suite, JMeter, nmap, OpenVAS, Scapy, Selenium ja SymCC.
        \item \textbf{RQ4:} -
    \end{itemize}

    Artikkelin vahvuuksia ovat kattava työkaluvertailu sekä standardiin sidottu, yksityiskohtainen CR-tason kattavuusanalyysi. Rajoituksiin kuuluvat mobiilikontekstin puuttuminen, kvantitatiivisten tulosten vähyys sekä teollisten järjestelmien erityispiirteisiin (kuten HMI-rajapintojen rajoitukset) liittyvät haasteet.
\end{description}

\begin{description}
    \item[\cite{aljohani2023_automating}] Artikkeli esittelee yhtenäisen kehyksen DevSecOps-prosessin automatisoidulle tietoturva-analyysille. Kehikko koostuu CI-putkeen liitettävästä agentista ja mikropalvelupohjaisesta moottorista, joka automatisoi SAST- ja DAST-skannaukset Shuffle-pohjaisessa SOAR-ympäristössä Docker-kontteja käyttäen.

    Tapaustutkimuksessa DVJA-verkkosovellus skannattiin GitHub Actions- ja Jenkins-putkissa Dependency-Checkin, Detect Secretesin ja Nikton avulla, ja koko suoritus vei 15 min 50 s, josta SAST 4 min 30 s (4 löydöstä) ja DAST 50 s (20 löydöstä)

    Tutkimus vastaa tutkimuskysymyksiin (RQ) seuraavasti:
    \begin{itemize}
        \item \textbf{RQ1:} -
        \item \textbf{RQ2:} Automaattiset työkalut nopeuttavat testausta ja vähentävät manuaalista työtä, mutta rajoituksina ovat väärien positiivisten tulosten määrä ja työkalujen konfiguroinnin vaativuus.
        \item \textbf{RQ3:} Artikkelissa mainitut työkalut ovat automatisoituja työkaluja: Dependency-Check (SAST), Detect Secrets (SAST) ja Nikto Scanner (DAST).
        \item \textbf{RQ4:} -
    \end{itemize}

    Tutkimuksen vahvuuksia ovat konkreettinen tapaustutkimus ja viitekehyksen käytännön sovellettavuuden osoittaminen, realistisen ympäristön käyttö testauksessa sekä kvantitatiivisten tulosten (suoritusajat, löydetyt haavoittuvuudet) esittäminen. Rajoituksia ovat rajattu teknologia (vain tietyt SAST- ja DAST-työkalut) sekä mobiilisovellusten sijaan yksittäisen Java-verkkosovelluksen testaus. Artikkelissa tunnistetaan myös Nikto-työkalun puute vakavuusluokituksessa rajoitteena.
\end{description}

\begin{description}
    \item[\cite{dupont2021_incremental}] esittelevät kevyen inkrementaalisen Common Criteria (CC) -sertifiointiprosessin integrointia DevSecOps-malliin tavoitteena automatisoida tietoturvasertifioinnin todisteiden keräämistä. Prosessia havainnollistetaan palomuurikomponentin (iptables/netfilter) päivityksellä autoalan platooning-järjestelmässä. Sertifiointiputkeen sisällytettiin SAST- ja DAST-työkalut sekä GitLab CI/CD- ja Jenkins-ympäristöt. Automatisoidulla lähestymistavalla vaikutusanalyysiraporttien (IAR) valmisteluaika lyheni merkittävästi ilman laadun heikkenemistä.

    Tutkimus vastaa tutkimuskysymyksiin (RQ) seuraavasti:
    \begin{itemize}
        \item \textbf{RQ1:} Integraatio toteutettiin mallintamalla CC- ja DevSecOps-vaiheet rinnakkain, parhaiksi käytännöiksi nousivat SARIF-standardin hyödyntäminen ja kaksivaiheinen raportointi.
        \item \textbf{RQ2:} Automatisointi nopeutti testejä ja vähensi virheitä, mutta monimutkaiset tapaukset vaativat yhä manuaalista tarkastusta.
        \item \textbf{RQ3:} Käytetyiksi automatisoinnin työkaluiksi nimetään SAST-työkalut Frama-C, SonarQube, Eclipse Steady, DAST-työkalut OpenSCAP, OpenVAS, OWASP ZAP, uhkamallinnustyökalut Threat Dragon ja Threagile sekä seurantaan ja operatiiviseen tietoturvaan tarkoitetut SIEM- ja SOAR-ratkaisut.
        \item \textbf{RQ4:} -
    \end{itemize}

    Artikkelin vahvuuksia ovat konkreettinen tapaustutkimus autoteollisuuden palomuurin päivityksestä, DevSecOps-prosessin mallintaminen aktiviteettidiagrammeilla  ja esitys siitä, miten automaattisesti kerättyjä todisteita (SAST, DAST) voidaan hyödyntää sertifioinnissa. Rajoituksia ovat, että tapaustutkimus keskittyy vain vaikutusanalyysivaiheeseen, mobiilisovelluskontekstin puute, ja vaikka työkaluja nimetään, kvantitatiivista dataa niiden tehokkuudesta tai vertailua eri työkalujen välillä ei esitetä yksityiskohtaisesti.
\end{description}

\begin{description}
\item[\cite{nutalapati2023_tools_techniques}] Artikkelissa tarkastellaan automatisoitua tietoturvatestausta mobiilisovelluksissa ja esitellään kattavasti käytettyjä työkaluja (mm. OWASP ZAP, Burp Suite ja Fortify) sekä tekniikoita (SAST, DAST, IAST). Tulosten mukaan automatisointi nopeuttaa haavoittuvuuksien tunnistamista,tarjoaa laajemman testauskattavuuden ja vähentää ihmislähtöisiä virheitä verrattuna manuaaliseen testaukseen. Haasteina mainitaan väärät positiiviset hälytykset, mobiiliympäristöjen monimutkaisuus ja kehittyvien uhkien jatkuva muuttuminen. Tutkimus vastaa alakysymyksiin RQ2 ja RQ3 kuvaamalla automatisoinnin etuja ja rajoituksia sekä esittelemällä konkreettisia testityökaluja. Laadunarvioinnissa vahvuuksina ovat kattava menetelmien esittely ja parhaat käytännöt. Rajoitteena mainitaan esitettyjen menetelmien yleisluonteisuus ilman yksittäisen tapaustutkimuksen tarjoamaa syvällisyyttä tai kvantitatiivista vertailua.
\end{description}

\begin{description}
    \item[\cite{saleem2023_survey}]
\end{description}

\begin{description}
    \item[\cite{chatterjee2024_operational_efficiency}]
\end{description}

\begin{description}
    \item[\cite{savor2016_facebook}]
\end{description}

\begin{description}
    \item[\cite{offerman2022_practices}]
\end{description}

\begin{description}
    \item[\cite{tony2019_practical}]
\end{description}

\begin{description}
    \item[\cite{rohin2018_book}]
\end{description}

\begin{description}
    \item[\cite{chung2024_devsecops}]
\end{description}

%----------------------------------------------------------------------
\section{RQ1 – Miten mobiilisovellusten tietoturvatestaus voidaan integroida osaksi jatkuvaa integraatiota ja toimitusprosessia DevSecOps-prosessissa, ja mitkä ovat tähän liittyvät parhaat käytännöt?}
\label{sec:rq1}

Tässä alaluvussa tarkastellaan n artikkelia, jotka kuvaavat
tietoturvatestauksen integrointia CI/CD-putkeen.

\subsection{Havainnot}
% TODO: kirjoita teemallinen/kronologinen synteesi + viittaukset

\subsection{Yhteenveto RQ1}
% TODO: 3–5 bulletia, joissa vastaus kysymykseen

%----------------------------------------------------------------------
\section{RQ2 – Mitkä ovat manuaalisen ja automatisoidun tietoturvatestauksen edut ja rajoitukset mobiilisovellusten DevSecOps-prosessissa?}
\label{sec:rq2}

\subsection{Lähdeviittaukset}
Marandi \emph{et al.}\ (\citeyear{marandi2023_ias}),
Rangnau \emph{et al.}\ (\citeyear{putra2022_devsecops}) ja
Nikolov \emph{et al.}\ (\citeyear{rangnau2020_cst}) raportoivat automatisoinnin
vaikutuksista manuaaliseen testaukseen (ks. liite~\ref{app:rq_matrix}).

\subsection{Havainnot}
% TODO

\subsection{Yhteenveto RQ2}
% TODO

%----------------------------------------------------------------------
\section{RQ3 – Mitä työkaluja käytetään automatisoituun ja manuaaliseen mobiilisovellusten tietoturvatestaukseen DevSecOps-prosessissa?}
\label{sec:rq3}

% TODO: taulukko työkaluista (SAST, DAST, uhkamallinnus, obfuskointi)
Ragnau \emph{et al.}\ (\citeyear{rangnau2020_cst}) ja
Feio \emph{et al.}\ (\citeyear{feio2024_empirical}) ovat lähteitä

\subsection{Automatisoidut työkalut}

\cite{feio2024_empirical} käy hyvin läpi eri työkaluja

\subsection{Manuaaliset työkalut}

\subsubsection{SAST}

\subsubsection{DAST}

\subsubsection{SCA}

%----------------------------------------------------------------------
\section{RQ4 – Miten DevSecOps-prosessi vaikuttaa mobiilisovellusten käyttöönottoon ja ylläpitoon?}
\label{sec:rq4}

% TODO: lyhyet havainnot (vähiten lähteitä)

\subsection{Käyttöönotto}

%----------------------------------------------------------------------
\section{Laadunarvio ja luotettavuus}
% TODO: CASP/JBI, bias-riski, lähteiden pisteytys

%----------------------------------------------------------------------
\section{Luvun yhteenveto}
% TODO: kiteytä vastaukset RQ1–RQ4 tiiviiksi listaksi

%======================================================================
\chapter{Analyysi}
\label{cha:analyysi}

Yhteenveto taulukoita

lisätään luvun \ref{cha:analyysi} taulukkoon sarake omalle tulkinnalle

Listataan tulokset muttei kerrota mitä ne tarkoittaa / analysoida / pohdiskella


\chapter{Yhteenveto}
\label{cha:yhteenveto}

Tehdään yöllä xd

Pyrittiin vastaamaan tähän ja tähän tutkimusongelmaan

päälöydökset

tutkimuskysymykset ja lyhyt vastaus jokaiseen

rajoitteet:
epätäydellisyyteni
mitä olisi voinut tehdä paremmin
olisko toinen aineisto ollut parempi

Jatkotukimusaiheet

Lorem ipsum dolor sit amet\citep{zafari2019survey}

\citet{zafari2019survey} Lorem ipsum dolor sit amet

%% Seuraavaksi tulee viiteluettelo
\printbibliography[heading=bibintoc]

\backmatter % Älä poista!
%% Mahdolliset liitteet tulevat tähän
\end{document}
\endinput
%%
%% End of file `minimal_modern.fi.tex'.
