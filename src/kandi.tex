%%
%% This is file `minimal_modern.fi.tex',
%% generated with the docstrip utility.
%%
%% The original source files were:
%%
%% uefcsthesis.dtx  (with options: `ex,fi,modern')
%% 
%% This is a generated file.
%% 
%% Copyright (C) 2018--2022 by Pauli Miettinen <pauli.miettinen@uef.fi>
%% 
%% This file may be distributed and/or modified under the conditions of
%% the LaTeX Project Public License, either version 1.3c of this license
%% or (at your option) any later version.  The latest version of this
%% license is in:
%% 
%% http://www.latex-project.org/lppl.txt
%% 
%% and version 1.3c or later is part of all distributions of LaTeX
%% version 2006/05/20 or later.
%% 
%% 
%% Tämä on yksinkertainen esimerkki uefcsthesis-luokan käytöstä.
%% Tämä tiedosto tuottaa pro gradu -tutkielman yksipuoleisella asettelulla.
%% Tuottaaksesi PDF-tiedoston, käytä joko lualatex- tai xelatex-ohjelmaa.
%% Esimerkiksi:
%% $ lualatex minimal_modern.fi.tex
%% $ biber minimal_modern.fi
%% $ lualatex minimal_modern.fi.tex
%% Vaihtoehtoisesti voit käyttää latexmk-ohjelmaa:
%% $ latexmk -lualatex minimal_modern.fi.tex
%%
%% Palautettaessa opinnäytetyö kirjastoon mukaan täytyy liittää lähdekoodi, josta
%% palautettava PDF on luoto. Helpointa tämä on jos lähdekoodi on ensin tiivistetty
%% yhteen tiedostoon (+ erilliset kuvatiedostot). Yhdistämiseen voi käyttää latexpand-
%% ohjelmaa (https://www.ctan.org/pkg/latexpand), joka tulee myös TeX Live- ja
%% MiKTeX-jakelupakettien mukana. Esimerkkikäyttö:
%% $ lualatex minimal_modern.en.tex
%% $ biber minimal_modern.en
%% $ latexpand --empty-comments --biber minimal_modern.en.bbl \
%% > minimal_modern.en.tex > flat_thesis.tex
%% $ lualatex flat_thesis
%% $ lualatex flat_thesis
%% joka tuottaa tiedostot flat_thesis.tex ja flat_thesis.pdf, jotka voidaan palauttaa
%% kirjastoon kuvatiedostojen kanssa.
%%
\documentclass[bscthesis,finnish,oneside,biblatex]{uefcsthesis}

%% Korvaa seuraavasta minimal.bib lähdeviitetietokantatiedostosi nimellä.
\addbibresource{library.bib}

%% Korvaa isolla kirjoitetut tekstit omilla tiedoillasi.
%% Työn otsikko, päiväys ja avainsanat täytyy antaa myös englanniksi
\title{Mobiilisovellusten tietoturvatestaus DevSecOps-prosessissa: Menetelmät, työkalut ja vaikutukset} % Työsi otsikko
\title[english]{Mobile Application Security Testing in the DevSecOps Process: Methods, Tools, and Implications} % Otsikko englanniksi
\author{Jesper}{Kauppinen} % Nimesi
\date{\thismonth} % Työsi valmistumiskuukausi ja -vuosi tai \thismonth automaattiseen päiväykseen
%% \date[english]{MONTH YEAR} % Englanninkielistä päiväystä ei tarvita, jos käytät \thismonth-komentoa, mutta se tarvitaan, jos kirjoitat päivän käsin
\city{Kuopio} % Joko Kuopio tai Joensuu
\firstsupervisor{Marko Jäntti} % Ensimmäisen ohjaajan nimi
\secondsupervisor{} % Toisen ohjaajan, jos on, nimi
\keywords{DevSecOps\sep Tietoturvatestaus\sep Mobiilisovellus} % Avainsanat erotetaan \sep-komennolla
\keywords[english]{DevSecOps\sep Security Testing\sep Mobile Application} % Avainsanat englanniksi

%% ACM:n CCS-luokittelun LaTeX-komennot saa luotua ACM:n työkalulla osoitteessa
%% https://dl.acm.org/ccs/ccs.cfm
%% Kopioi työkalun tuottama LaTeX-koodi tähän (alun XML-koodia ei tarvitse
%% kopioida). Esimerkiksi:
%% \ccsdesc[500]{Some Class}

\begin{document}
\maketitle
\begin{abstract}
KIRJOITA SUOMENKIELINEN TIIVISTELMÄSI TÄHÄN
\end{abstract}

\begin{abstract}[english]
KIRJOITA ENGLANNINKIELINEN TIIVISTELMÄSI TÄHÄN
\end{abstract}

\frontmatter
\tableofcontents
\mainmatter

\chapter{Johdanto}
\label{cha:johdanto}

Tutkimuskysymykset:

\begin{description}
    \item[Pääkysymys:] MMiten mobiilisovellusten tietoturvatestaus voidaan integroida osaksi jatkuvaa integraatiota ja toimitusprosessia DevSecOps-prosessissa, ja mitkä ovat tähän liittyvät parhaat käytännöt?
    \begin{description}
        \item[Alakysymys 1:] Mitkä ovat manuaalisen ja automatisoidun tietoturvatestauksen edut ja rajoitukset mobiilisovellusten DevSecOps-prosessissa?
        \item[Alakysymys 2:] Mitä työkaluja käytetään automatisoituun ja manuaaliseen mobiilisovellusten tietoturvatestaukseen DevSecOps-prosessissa?
        \item[Alakysymys 3:] Miten DevSecOps-prosessi vaikuttaa mobiilisovellusten käyttöönottoon ja ylläpitoon?
    \end{description}
\end{description}

Biblio Authoreiden siivous:
Etunimen eka kirjain -> etunimi

Tutkimuksen rakenne

Johdannon alku:
motivoidaan lukija
Muut aikaisemmat tutkimukset: miten 5 aikaisempa artikkelia, viittauksia lähteisiin
Miksi tutkimusta tarvitaan? Ei aikaisempaa tutkimusta tähän specifiin aiheeseen
minun työn tavoitteet: pääcontribuutio
Työn rakenne: mitä luvut käsittelevät (1 kappale)

\chapter{Teoria}
\label{cha:teoria}
Määritellään käsitteet DevSecOps, CI/CD, tietoturvatestaus, mobiilisovellus, jne.

\chapter{Menetelmät}
\label{cha:menetelmat}
Kuinka aineistoja analysoidaan, mitä aineistoja käytetty
listataan hakualauseet, joilla aineistoja haettu, include ja exclude
Valitut aineistot: noin 30 artikkelia

\chapter{Tulokset}
\label{cha:tulokset}

Kirjallisuuskatsaus löydökset taulukkona

Mitä artikkeleissa sanottiin...

\section{Miten mobiilisovellusten tietoturvatestaus voidaan integroida osaksi jatkuvaa integraatiota ja toimitusprosessia DevSecOps-prosessissa, ja mitkä ovat tähän liittyvät parhaat käytännöt? }

Miten mobiilisovellusten tietoturvatestaus voidaan integroida osaksi jatkuvaa integraatiota ja toimitusprosessia DevSecOps-prosessissa, ja mitkä ovat tähän liittyvät parhaat käytännöt?

\section{Mitkä ovat manuaalisen ja automatisoidun tietoturvatestauksen edut ja rajoitukset mobiilisovellusten DevSecOps-prosessissa? }

Mitkä ovat manuaalisen ja automatisoidun tietoturvatestauksen edut ja rajoitukset mobiilisovellusten DevSecOps-prosessissa?

\section{Mitä työkaluja käytetään automatisoituun ja manuaaliseen mobiilisovellusten tietoturvatestaukseen DevSecOps-prosessissa? }

Mitä työkaluja käytetään automatisoituun ja manuaaliseen mobiilisovellusten tietoturvatestaukseen DevSecOps-prosessissa?

\subsection{Automatisoidut työkalut}

\subsection{Manuaaliset työkalut}

\subsubsection{SAST}

\subsubsection{DAST}

\subsubsection{SCA}

\section{Miten DevSecOps-prosessi vaikuttaa mobiilisovellusten käyttöönottoon ja ylläpitoon? }

Miten DevSecOps-prosessi vaikuttaa mobiilisovellusten käyttöönottoon ja ylläpitoon?

\subsection{Käyttöönotto}

\chapter{Analyysi}
\label{cha:analyysi}

Yhteenveto taulukoita

lisätään luvun \ref{cha:analyysi} taulukkoon sarake omalle tulkinnalle

Listataan tulokset muttei kerrota mitä ne tarkoittaa / analysoida / pohdiskella


\chapter{Yhteenveto}
\label{cha:yhteenveto}

Tehdään yöllä xd

Pyrittiin vastaamaan tähän ja tähän tutkimusongelmaan

päälöydökset

tutkimuskysymykset ja lyhyt vastaus jokaiseen

rajoitteet:
epätäydellisyyteni
mitä olisi voinut tehdä paremmin
olisko toinen aineisto ollut parempi

Jatkotukimusaiheet

Lorem ipsum dolor sit amet\citep{zafari2019survey}

\citet{zafari2019survey} Lorem ipsum dolor sit amet

%% Seuraavaksi tulee viiteluettelo
\printbibliography[heading=bibintoc]

\backmatter % Älä poista!
%% Mahdolliset liitteet tulevat tähän
\end{document}
\endinput
%%
%% End of file `minimal_modern.fi.tex'.
